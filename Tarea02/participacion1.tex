\documentclass[12pt]{article}
\usepackage{graphicx}
\usepackage{caption}
\usepackage{subcaption}
\usepackage{tikz}
\usepackage{venndiagram}
\usepackage{venndiagram}
\usepackage{tcolorbox}
\usepackage{listings}
\usepackage{enumitem}
\usepackage{amsmath}
\usepackage{amssymb}
\usepackage{colortbl}
\usepackage{xcolor}
\usepackage[margin=1cm, top=1.5cm, bottom=1.5cm]{geometry}

\tcbuselibrary{breakable}

\title{\textbf{Álgebra Superior I: Tarea 01}}
\author{Rendón Ávila Jesús Mateo}
\date{\today}

\begin{document}

\maketitle
\begin{center}
\vspace{3cm}
\includegraphics[width=0.195\textwidth]{Escudo.png}
\hspace{0.5cm}
\includegraphics[width=0.2\textwidth]{logo_ciencias.png}
\end{center}
\begin{center}
    \vspace{1cm}
    Universidad Nacional Autónoma de México\\
    Facultad de Ciencias\\
    Profesora: Cristina Angélica Núñez Rodríguez\\
\end{center}

\newpage

%
% Ejercicio 1
%
\textbf{1.} Determinar qué propiedades (reflexividad, simetría, antisimetría o transitividad) cumplen las siguientes
relaciones y determinar cuáles son una relación de equivalencia o de orden (parcial o total).

\begin{enumerate}[label=\alph*)]
    \item $R = \{(x,y) \in \mathbb{R} \times \mathbb{R} | x-y \textit{ es múltiplo de 3}\}$

    \textit{Reflexividad}. Como $0$ es múltiplo de cualquier número y ademas $\forall x \in \mathbb{R}$ se cumple que $x-x = 0$, entonces $R$ satisface reflexividad.\\

    \textit{Simetría}. Si $(x-y)$ es un múltiplo de $3$, entonces $(y-x)$ también será múltiplo de $3$, en particular el inverso de $(x,y)$. De lo
    anterior decimos que $R$ satisface la simetría.\\

    \textit{Antisimetría}. La antisimetría no se cumple en $R$, basta dar el contraejemplo $(3,6)$ y $(6,3)$ donde $3 \neq 6$.\\

    \textit{Transitividad}. Finalmente, la relación satisface la transitividad pues $\forall$ $(x-y), (y-z)$ que es multiplo de $3$, también el número $(x-z)$ satisface
    el ser múltiplo de 3.\\

    Por lo tanto $R$ es de equivalencia.

    \item $R = \{(1,1),(2,2),(1,2),(2,1),(3,3),(3,4),(4,3),(4,4)\}$, donde $A = \{1,2,3,4\}$

    \textit{Reflexividad}. Como $\forall x \in A, \text{ } \exists (x,x) \in R$ decimos que R es reflexiva.\\ 

    \textit{Simetría}. Como $(1,2), (3,4) \in R$  y $(2,1), (4,3) \in R$ entonces R es simétrica, para los pares $(x,x)$ la simetría es por
    vacuidad.\\

    \textit{Antisimetría}. No se satisface.

    \textit{Transitividad}. 
    \begin{align*}
        (1,1), (1,2)&\sim (1,2)\\
        (2,2), (2,1)&\sim (2,1)\\
        (1,2), (2,1)&\sim (1,1)\\
        (2,1), (1,2)&\sim (2,2)\\
    \end{align*}
    Para el caso de $3$ y $4$ es similar, así $R$ es transitiva.\\
    
    Por lo tanto $R$ es de equivalencia.

    \item $R = \{(1,1),(2,2),(3,3),(4,4),(1,2),(1,3),(1,4),(2,3)\}$, donde $A = \{1,2,3,4\}$

    \textit{Reflexividad}. Como $\forall x \in A, \text{ } \exists (x,x) \in R$ decimos que R es reflexiva.\\ 

    \textit{Simetría}. Como $(1,2) \in R$  y $(2,1) \notin R$, entonces R no es simétrica.\\

    \textit{Antisimetría}. No se satisface.

    \textit{Transitividad}. El caso trascendente es que $\exists (1,2), (2,3) \in R$ y tmabién $(1,3) \in R$\\

    Por lo tanto R no es de equivalencia.\\

    \item La relación en $A = \mathbb{R}$ definida por $a \sim b \Longleftrightarrow a \leq b$
    \item La relación en $A = P(X)$ definida por $A \sim B \Longleftrightarrow A \subseteq B$
\end{enumerate}

\vspace{0.5cm}
%
% Ejercicio 2
%
\textbf{2.} Demostrar que la siguiente relación es de equivalencia e indicar quién es el conjunto cociente asociado. 
Sea $A = \{(a, b) \mid a, b \in \mathbb{Z}, b \neq 0\}$ y $R$ la relación definida en $A$ tal que $(a, b) \sim (c, d)$ si y sólo si $ad = bc$.\\

$P.d$ $R$ es reflexiva, simétrica y transitiva.\\

\textit{R es reflexiva.}\\

$\forall (a, b) \in A$ se debe satisfacer que $(a, b) \sim (a, b)$
\begin{align*}
    (a, b) &\sim (a, b)\\
    ab &= ba \textit{ (por conmutación en ba)}\\
    ab &= ab
\end{align*}
Así, decimos que $R$ es reflexiva.\\

\textit{R es simétrica.}\\

Si $(a, b) \sim (c, d)$ entonces se deberá satisfacer que $(c, d) \sim (a, b)$
\begin{align*}
    (a, b) &\sim (c, d)\\
    ad &= cb \\
    cb &= ad \textit{ (por conmutación en ad)}\\ 
    cb &= da\\
    (c, d) &\sim (a, b)
\end{align*}
Así, $R$ es simétrica.\\

\textit{R es transitiva.}\\

Si $(a, b) \sim (c, d)$ y $(c, d) \sim (e, f)$ entonces se deberá satisfacer que $(a, b) \sim (e, f)$
\begin{align*}
    (a, b) &\sim (c, d)\\
    ad &= bc\\
    \\
    (c, d) &\sim (e, f)\\
    cf &= de\\
\end{align*}
Por propiedades de los $\mathbb{Z}$
\begin{align*}
    ad \bullet cf &= bc \bullet de \textit{ (por cancelación del producto en $\mathbb{Z}$)}\\
    af &= be\\
    (a, b) &\sim (e, f)
\end{align*}
Así concluimos que $f$ es transitiva.\\
\\
$\therefore$ $R$ es de equivalencia.

\vspace{0.5cm}
%
% Ejercicio 3
%
\textbf{3.} Diga cuál de las siguientes relaciones son funciones (justifica tu respuesta):
\begin{enumerate}[label=\alph*)]
    \item $R \subseteq \{1,2,3\} \times \{1,2,3\}$ definida como:\\
    \begin{center}
        $R = \{(1,2),(2,2),(3,3),(2,3),(1,1)\}$
    \end{center}
    No es función pues $1$ está relacionado con $1$ y $2$ y también $2$ está relacionada con $2$ y $3$.

    \item $S \subseteq \mathbb{N} \times \mathbb{N}$ definida como:\\
    \begin{center}
        $S = \{(n,m) \mid n < m\}$
    \end{center}
    Es facil ver que si $n$ está en el dominio de $S$ y $n = 1$, entonces para cualquier $m > 1$ en el codominio tendremos que $n = 1$
    va a satisfacer el estar relacionado con $m > 1$, con lo que $S$ no es función.
    
    \item $T \subseteq ( \mathbb{Z} \times \mathbb{Z} ) \times \mathbb{Z}$ definida como:\\
    \begin{center}
        $T = \{((n,m),n+m) \mid n, m \in \mathbb{Z}\}$
    \end{center}
    Como no puede ser que $m \neq m$ o $n \neq n$, el valor para $n + m$ debe ser unico.\\

    Pensemos en una $m'$ que satisfaga $n + m = n + m'$
    \begin{align*}
        n + m &= n + m'\\
        m &= m'
    \end{align*}
    Lo mismo ocurre con una $n'$ por lo que $n + m$ es un valor único.\\

    $\therefore$ $f$ es función pues $n + m$ es un valor unico para cualquier $n, m \in \mathbb{Z}$.

\end{enumerate}

\vspace{0.5cm}
%
% Ejercicio 4
%
\textbf{4.} Sea ${\displaystyle f: \mathbb{Q} \longrightarrow \mathbb{Z}}$, con regla de correspondencia:\\
\begin{center}
    ${\displaystyle f(\frac{a}{b}) = a}$
\end{center}
Para toda $a.b \in \mathbb{Z}$, con $b \neq 0$. ¿Será que ${\displaystyle f}$ está bien definida?. Justifica tu respuesta.

\vspace{0.5cm}
%
% Ejercicio 5
%
\textbf{5.} Sea ${\displaystyle f: A \longrightarrow B}$ y sean $Y_1, Y_2 \subseteq B$. Demuestra lo siguiente:
\begin{enumerate}[label=\alph*)]
    \item ${\displaystyle f^{-1}} [\varnothing] = \varnothing$
    \item ${\displaystyle f^{-1}} [Y_1 \cup Y_2] = {\displaystyle f^{-1}}[Y_1] \cup {\displaystyle f^{-1}}[Y_2]$
    \item ${\displaystyle f^{-1}} [Y_1 \cap Y_2] = {\displaystyle f^{-1}}[Y_1] \cap {\displaystyle f^{-1}}[Y_2]$
    \item ${\displaystyle f^{-1}} [B] \backslash {\displaystyle f^{-1}} [Y_1] = {\displaystyle f^{-1}}[B \backslash Y_1]$
\end{enumerate}

\vspace{0.5cm}
%
% Ejercicio 6 
%
\textbf{6.} Da un contraejemplo de una función ${\displaystyle f: A \longrightarrow B}$ y $X_1, X_2 \subseteq A$ tales que:
\begin{center}
    ${\displaystyle f} [X_1 \cap X_2] \neq {\displaystyle f}[X_1] \cap {\displaystyle f}[X_2]$
\end{center}

\vspace{0.5cm}
%
% Ejercicio 7
%
\textbf{7.} Sean ${\displaystyle f: A \longrightarrow B}$ y ${\displaystyle g: B \longrightarrow C}$ funciones. Demuestre lo siguiente:
\begin{enumerate}[label=\alph*)]
    \item si ${\displaystyle f}$ y ${\displaystyle g}$ son inyectivas, entonces ${\displaystyle g \circ f}$ es inyectiva
    \item si ${\displaystyle f}$ y ${\displaystyle g}$ son sobreyectivas, entonces ${\displaystyle g \circ f}$ es sobreyectiva
    \item si ${\displaystyle f}$ y ${\displaystyle g}$ son biyectivas, entonces ${\displaystyle g \circ f}$ es biyectiva.
\end{enumerate}

\vspace{0.5cm}
%
% Ejercicio 8
%
\textbf{8.} Sean ${\displaystyle f: A \longrightarrow B}$ y ${\displaystyle g: B \longrightarrow C}$ funciones invertibles.
\begin{enumerate}[label=\alph*)]
    \item Demuestre que ${\displaystyle g \circ f}$ es invertibles
    \item Demuestre que ${\displaystyle (g \circ f)^{-1}} = {\displaystyle f^{-1} \circ g^{-1}}$ 
\end{enumerate}

\vspace{0.5cm}
% 
% Ejercicio 9
%
\textbf{9.} Sea ${\displaystyle f: A \longrightarrow B}$ inyectiva. Demostrar que si $B$ es finito entonces $A$ es finito y $\# A \leq \#B$

\vspace{0.5cm}
%
% ejercicio 10
%
\textbf{10.} Sea ${\displaystyle f: A \longrightarrow B}$ suprayectiva. Demostrar que si $A$ es finito entonces $B$ es finito y $\# B \leq \#A$

\end{document}