\documentclass[12pt]{article}
\usepackage{graphicx}
\usepackage{caption}
\usepackage{subcaption}
\usepackage{tikz}
\usepackage{tcolorbox}
\usepackage{listings}
\usepackage{enumitem}
\usepackage{amsmath}
\usepackage{amssymb}
\usepackage{xcolor}
\usepackage[margin=1cm, top=1.5cm, bottom=1.5cm]{geometry}

\tcbuselibrary{breakable}

\title{\textbf{Álgebra Superior I: Tarea 01}}
\author{Rendón Ávila Jesús Mateo}
\date{\today}

\begin{document}

\maketitle
\begin{center}
\vspace{3cm}
\includegraphics[width=0.195\textwidth]{Escudo.png}
\hspace{0.5cm}
\includegraphics[width=0.2\textwidth]{logo_ciencias.png}
\end{center}
\begin{center}
    \vspace{1cm}
    Universidad Nacional Autónoma de México\\
    Facultad de Ciencias\\
    Profesora: Cristina Angélica Núñez Rodríguez\\
\end{center}

\newpage

%
% Ejercicio 1
%
\textbf{1.} Encuentra una proposición adecuada para describir a cada uno de los siguientes conjuntos.
\begin{enumerate}[label=\alph*)]
    \item $A = \{30, 31, 32, ...\}$\\
    $A = \{x \in \mathbb{N} \mid x \geq 30\}$
    \item $B = \{-1, 2, -3, 4, -5, 6, -7, ...\}$\\
    $B = \{x \in \mathbb{N}-\{0\} \mid \text{ si } x \mod 2 \neq 0 \Rightarrow -1(x)\}$
    \item $C = \{-1, 3, -5, 7, -9, 11, ...\}$\\
    $C = \{x \in \mathbb{N} - \{0\} \mid \text{ si } \}$
    \item $D = \{4, 7, 12, 19, ...\}$\\
    $D = \{...\}$
\end{enumerate}

%
% Ejercicio 2
%
\textbf{2.} Describe los siguientes conjuntos listando todos sus elementos.
\begin{enumerate}[label=\alph*)]
    \item $\{x \in \mathbb{N} \mid x^2 - 3x = 0\}$\\
    $\{3\}$

    \item $\{n^3 + n^2 \mid n \in \{0, 1, 2, 3, 4\}\}$\\
    $\{0, 2, 12, 36, 80\}$

    \item $\{\frac{1}{n^2 + n}\mid n \text{ es un positivo impar y } n \in \{1, 2, 3, 4, 5, 7\}\}$\\
    $\{\frac{1}{2}, \frac{1}{12}, \frac{1}{30}, \frac{1}{56}\}$

    \item $\{l \in \mathbb{Z} \mid l = 2n -1 \text{ y } -3 \leq n \leq 9\}$\\
    $\{-7, -5, -3, -1, 1, 3, 5, 7, 9, 11, 13, 15, 17\}$
\end{enumerate}

%
% Ejercicio 3
%
\textbf{3.} Sea $\mathbb{N}$ el conjunto de los números naturales. Determinar $ A \cup B$, $A \cap B$ y $A^c$ en:
\begin{enumerate}[label=\alph*)]
    \item $A = \{n \mid n \text{ es par}\}$ y $B = \{n \mid n < 14\}$\\
    $A \cup B = \{2, 4, 6, 8, 10, ...\}$\\
    $A \cap B = \{2, 4, 6, 8, 10, 12\}$\\
    $A^c = \{1, 3, 5, 7, 9, 11, ...\}$

    \item $A = \{n \mid n^2 > 2n - 1\}$, $B = \{n \mid n^2 = 2n + 3\}$\\
    $A \cup B =  \{2, 3, 4, 5, 6, ...\}$\\
    $A \cap B = \{9\}$\\
    $A^c = \{0, 1\}$
\end{enumerate}

%
% Ejercicio 4
%
\textbf{4.} Dibuja el diagrama de Venn para el siguiente problema:
\\

Un grupo de jóvenes fue entrevistado acerca de sus preferencias por diferentes medios
de transporte: bicicleta, motocicleta y automóvil. Los datos de las encuestas fueron los
siguientes:
\begin{itemize}
    \item Motocilceta solamente 5.
    \item Motocicleta 38.
    \item No gustan del automóvil 9.
    \item Motocicleta y bicicleta pero no automóvil 3.
    \item Motocicleta y automóvil pero no bicicleta 20.
    \item No gustan de la bicicleta 72.
    \item No gustan de las tres cosas 19.
    \item No gustan de la motocicleta 61.
\end{itemize}

Determinar: 

\begin{enumerate}[label=\alph*)]
    \item ¿Cuál fue el número de personas entrevistadas?\\
    99 personas
    \item ¿A cuántos les gusta la bicicleta solamente?\\
    4 personas
    \item ¿A cuántos les gusta el automóvil solamente?\\
    34 personas
    \item ¿A cuántos les gustan las tres cosas?\\
    10 personas
    \item ¿A cuántos les gusta la bicicleta y el automóvil pero no la motocicleta?\\
    4 personas
\end{enumerate}

%
% Ejercicio 5
%
\textbf{5.} Sean $A$ y $B$ conjuntos. Prueba que:
\begin{enumerate}[label=\alph*)]
    \item $A - B = B^c - A^c$

    $A - B \Longleftrightarrow A \cap B^c$ \textit{por propiedad de la diferencia}\\
    $A \cap B^c \Longleftrightarrow B^c \cap A$ \textit{por conmutatividad de la intersección}\\
    $B^c \cap A \Longleftrightarrow B^c - A^c$ \textit{por propiedad de la diferencia}

    \item $B \subseteq A \Longleftrightarrow (A - B) \cup B = A$

    $\subseteq)$ Sea $x \in B$ por definición de subconjunto $x \in A$\\
    como $x \in A$ y $x \in B$ entonces $x \notin A - B$\\
    como $x \notin A - B$ y $x \in B$ entonces $x \in (A - B) \cup B$\\
    y por hipotesis, como $B \subseteq A$ entonces $B \cup (A - B) = A$\\
    \\
    $\supseteq)$ Sea $x \in (A - B) \cup B$ entonces\\
    $x \in A - B$ o $x \in B$\\
    como por nuestra hipotesis $(A - B) \cup B = A$ entonces $x \in A$\\
    por lo que debe ser $x \in A - B$
    como $x \notin B$ pero si $x \in (A - B) \cup B$ entonces debe ser $B \subseteq A$\\
    \\
    $\therefore B \subseteq A \Longleftrightarrow (A - B) \cup B = A$
\end{enumerate}

%
% Ejercicio 13
%
\textbf{13.} Determina cuáles de las siguientes oraciones son proposiciones:

\begin{enumerate}[label=\alph*)]
    \item Algunos números enteros son negativos.\\
    Es una proposición con valor de verdad $V$.
    \item El número 15 es un número par.\\
    Es una proposición con valor de verdad $F$.
    \item ¿Qué hora es?\\
    No es una proposición pues no se puede determinar veracidad.
    \item En los números enteros, $11 + 6 \neq 12$\\
    Si es proposición con valor de verdad $V$.
    \item La tierra es casi una esfera.\\
    Si es una proposición con valor de verdad $V$.
\end{enumerate}

%
% Ejercicio 14
%
\textbf{14.} Si $P$ y $R$ representan proposiciones verdaderas y $Q$ y $S$ representan proposiciones falsas,
encuentra el valor de verdad de las proposiciones compuestas dadas a continuación:

\begin{enumerate}[label=\alph*)]
    \item $\neg P \land R$\\
    $F \land V = F$

    \item $\neg [\neg P \land (\neg Q \land P)]$\\
    $\neg[F \land (V \land V)] = \neg[F \land V] = \neg F = V$

    \item $(P \land R) \lor \neg Q$\\
    $(V \land V) \lor V = V \lor V = V$

    \item $P \Longrightarrow (Q \Longrightarrow R)$\\
    $V \Longrightarrow (F \Longrightarrow V) = V \Longrightarrow V = V$

    \item $[(P \land \neg Q) \Longrightarrow (Q \land R)] \Longrightarrow (S \lor \neg Q)$\\
    $[(V \land V) \Longrightarrow (F \land V)] \Longrightarrow (F \lor V) = [V \Longrightarrow F] \Longrightarrow V = F \Longrightarrow V = V$
\end{enumerate}

%
% Ejercicio 15
%
\textbf{15.} Responde:

\begin{enumerate}[label=\alph*)]
    \item Si la proposición $Q$ es verdadera, determine todas las asiganciones de valores de
    verdad para las proposiciones $P$, $R$ y $S$ para la proposición:
    \[\{Q \Longrightarrow [(\neg P \lor R) \land (\neg S)]\} \land \{\neg S \Longrightarrow (\neg R \land Q)\}\]
    \begin{table}[h!]
        \centering
        \begin{tabular}{|c|c|c|c|c|c|c|c|}
            \hline
            $Q$ & $P$ & $R$ & $S$ & $\neg P$ & $\neg R$ & $\neg S$\\
            \hline
            V & V & V & V & F & F & F\\
            V & V & V & F & F & F & V\\
            V & V & F & V & F & V & F\\
            V & V & F & F & F & V & V\\
            V & F & V & V & V & F & F\\
            V & F & V & F & V & F & V\\
            V & F & F & V & V & V & F\\
            V & F & F & F & V & V & V\\ 
            \hline           
        \end{tabular}
        \caption{Tabla de valor de las variables}
    \end{table}
    \newpage

        \item Lo mismo que en a), pero suponiendo que $Q$ es falsa.\\
        \begin{table}[h!]
            \centering
            \begin{tabular}{|c|c|c|c|c|c|c|c|}
                \hline
                $Q$ & $P$ & $R$ & $S$ & $\neg P$ & $\neg R$ & $\neg S$\\
                \hline
                F & V & V & V & F & F & F\\
                F & V & V & F & F & F & V\\
                F & V & F & V & F & V & F\\
                F & V & F & F & F & V & V\\
                F & F & V & V & V & F & F\\
                F & F & V & F & V & F & V\\
                F & F & F & V & V & V & F\\
                F & F & F & F & V & V & V\\ 
                \hline           
            \end{tabular}
            \caption{Tabla de valor de las variables}
        \end{table}
\end{enumerate}

\end{document}